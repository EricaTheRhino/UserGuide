\chapter{Erica Event Setup}
This covers connections that are required for Erica to operate at events.

\section{Important details about Erica}
\begin{description}
	\item[Power use]{\dots watts (\dots A)}
	\item[Physical dimensions (on plinth)]{2500x1000x1500mm (LxWxH)}
\end{description}


\section{External Connections}
Erica's plinth has two connection panels, one for power and the other for data.


\subsection{Power Connection Panel}
Erica's power connection plate is located on the right of the rear of the plinth.


\subsubsection{Fusing}
Erica is fused at 13A, A fuse no larger than 13A should be used. Smaller values as low as \dots may be used if a 13A fuse is unavailable although this will limit the power that can be drawn from the plinth power out.


\subsubsection{Power Connections}
Erica is designed for use with 240V AC 50Hz power systems. Erica must not be connected to other voltage or frequency power systems. 

Erica has two Neutrik powerCON connectors for power. The bottom blue connector is for power in, Erica is fused internally at 13A so powerdraw will not be more than this.  The upper grey connector is power out, this is capable of a little less than \dots A, as it shares the same fuse with the electronics inside Erica.


\subsection{Data Connection Panel}
Erica's data connection plate is located on the left of the rear of the plinth.

\begin{table}[h]
\centering
\begin{tabular}[h]{|c|c|c|c|}
\hline
Left & 6.35mm & HDMI from & External \\
Audio XLR & Audio Jack & Interface & Ethernet in \\
\hline
Right & Blank & USB to & Internal \\
Audio XLR & & Interface & Ethernet out \\
\hline
\end{tabular}
\caption{Layout of Data Connection Panel} 
\end{table}


\subsubsection{Audio Out}
The two XLR connections provide line level balanced audio out for connection to balanced audio systems.  The upper XLR connector is the left channel the lower connector is the right channel.

The 6.35mm audio jack provides line level unbalanced stereo audio out.

\subsubsection{HDMI Out}
This is a Neutrik HDMI connector, it will accept either a Neutrik ruggedised HDMI connector or standard HDMI connector.  The HDMI connection is connected to the interface Raspberry Pi, this is generally only intended for debugging.

\subsubsection{USB In}
This is a Neutrik USB connector, it will accept either Neutrik ruggedised USB connector or standard USB connectors.  This is connected internally to the interface pi, this is intended to attach a keyboard during debugging.

\subsubsection{Network Connections}
These are Neutrik Cat 5 EtherCON connectors, these will accept RJ45 terminated Cat 5/6 cables with or without Cat 5 EtherCON housings. 
The Cat 6 version of EtherCON connector is incompatible.

The upper connection is for External network connectivity.  This port is capable of 10/100 MBit operation. This connection should offer IPv4 addresses over DHCP and provide Internet connectivity.

The lower connection connects to Erica's internal network.  This port is capable of 10/100/1000 MBit operation.  This is intended for connection external components so they can access Erica's internal network.


\section{Internal Connections}
When Erica is in transit the Pi board that has the five Raspberry Pis and 8 port Ethernet hub attached should be disconnected and removed.  When the Pi board is reconnected the following connections need to be made.

\subsection{Power}
5V and 12V

\subsection{Network}

\subsection{HDMI}

\subsection{USB}

\subsection{Audio}

\subsection{LEDs}

\subsection{Eyes}

\subsection{Ears}
